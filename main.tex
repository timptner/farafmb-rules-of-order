\documentclass[%
	parskip=half,
]{scrartcl}

\usepackage{microtype}

% Deutsche Rechtschreibung
\usepackage[ngerman]{babel}

% Listen mit Buchstaben
\usepackage{enumerate}
\newcounter{enumitem}

% Kunstvolle Brüche
\usepackage{nicefrac}

\usepackage[%
	clausemark=forceboth,
	juratotoc,
	juratocnumberwidth=2.5em,
]{scrjura}

\pagestyle{myheadings}

% Auswählbare Verlinkungen
\usepackage[hidelinks]{hyperref}

\useshorthands{'}
\defineshorthand{'S}{\Sentence\ignorespaces}
\defineshorthand{'.}{. \Sentence\ignorespaces}

% Fußnoten mit römischen anstatt arabischen Zahlen
\renewcommand{\thefootnote}{\roman{footnote}}

% Erweiterte Formatierung von Tabellen
\usepackage{booktabs}

\begin{document}

\title{Geschäftsordnung}
\date{27.\,06.\,2022}
\author{Fachschaftsrat Maschinenbau}

\maketitle

Geschäftsordnung für den Fachschaftsrat Maschinenbau der Otto-von-Guericke-Universität Magdeburg - nachfolgend
Fachschaftsrat genannt - in der geänderten und beschlossenen Fassung vom 27.06.2022.

\tableofcontents

\clearpage

\addsec{Begriffsbestimmungen}

Zum besseren Verständnis der Geschäftsordnung sind nachfolgend Erläuterungen zu den verwendeten Formulierungen zu 
finden.

\minisec{Gewählte Mitglieder}

Gewählte Mitglieder sind Studentinnen und Studenten, die während der Hochschulwahlen direkt gewählt wurden oder indirekt
aufgrund des Rücktritts eines Mitgliedes nachfolgen. Sie sind stimmberechtigt. Bei Beendigung der Mitgliedschaft eines
gewählten Mitgliedes wird ein stellvertretendes Mitglied zum Nachfolger. Hierbei ist die Reihenfolge entsprechend der
amtlichen Wahlergebnisse zu berücksichtigten.

\minisec{Stellvertretende Mitglieder}

Stellvertretende Mitglieder sind Studentinnen und Studenten, die aufgrund der erreichten Stimmenzahl während der
Hochschulwahlen als Stellvertretung gewählt wurden. Jedes stellvertretende Mitglied kann ein nicht anwesendes gewähltes
Mitglied auf den Sitzungen für die Dauer einer Sitzung vertreten und erhält hierdurch das Stimmrecht jenes Mitglieds.
Dabei ist die Reihenfolge der Vertretungen festgelegt entsprechend der erreichten Stimmenzahl während der
Hochschulwahlen. Eine Stellvertretung kann pro Sitzung nur ein gewähltes Mitglied vertreten.

\minisec{Kooptierte Mitglieder}

Kooptierte Mitglieder sind Studentinnen und Studenten, die durch Kooptation in den Fachschaftsrat aufgenommen wurden.
Sie sind nicht stimmberechtigt.

\minisec{Ehrenmitglieder}

Ehrenmitglieder sind ehemalige Mitglieder des Fachschaftsrates, welche nicht länger Studentin oder Student an der 
Fakultät Maschinenbau sind, jedoch aufgrund ihrer überdurchschnittlichen und langfristigen Mitwirkung im Fachschaftsrat 
zum Ehrenmitglied ernannt wurden. Sie sind nicht stimmberechtigt.

\minisec{Textform}

Bezeichnet das Verfassen eines (physischen oder digitalen) Dokuments, worin der Verfasser erkenntlich ist. Eine
Unterschrift wird nicht benötigt.

\minisec{Elektronische Form}

Bezeichnet das Verfassen eines digitalen Dokuments mit digitaler Signatur der erstellenden Person(en).

\minisec{Schriftform / schriftlich}

Bezeichnet das Verfassen eines physischen Dokuments mit Unterschrift der erstellenden Person(en).

\clearpage

\appendix

\section{Aufbau}

\begin{contract}

\Clause{title={Anwendungsbereich}}

Die Geschäftsordnung des Fachschaftsrates regelt insbesondere:
\begin{enumerate}[\qquad a)]
	\item den Ablauf, die Organisation und Durchführung der Sitzungen;
	\item die Zusammensetzung;
	\item sowie die Arbeit und interne Struktur.
\end{enumerate}

\Clause{title={Zusammensetzung}}

'S Der Fachschaftsrat besteht aus aktiven und passiven Mitgliedern'. Als aktive Mitglieder werden gewählte,
stellvertretende und kooptierte Mitglieder bezeichnet. Passive Mitglieder sind Ehrenmitglieder'. Die Anzahl der
gewählten Mitglieder beträgt 7'. Jedes aktive Mitglied des Fachschaftsrates muss auch Mitglied der Fachschaft
Maschinenbau an der Otto-von-Guericke-Universität Magdeburg sein'. Die Kooptation von aktiven Mitgliedern wird mit einer
\nicefrac{2}{3}-Mehrheit der stimmberechtigten Mitglieder beschlossen.

Der Fachschaftsrat teilt seine Arbeit in die folgenden Sachgebiete ein:
\begin{enumerate}[\qquad a)]
	\item Verwaltung;
	\item Finanzen;
	\item Öffentlichkeitsarbeit;
	\item Studium \& Lehre;
	\item Exkursionen;
	\item Hochschulpolitik;
	\item Mentoring;
	\item Technik;
	\item Veranstaltungen.
\end{enumerate}

\Clause{title={Vorstand}}

'S Der Fachschaftsrat wählt aus seiner Mitte Sprecherinnen und Sprecher, welche gemeinsam den Vorstand bilden'. Der
Fachschaftsrat kann von mindestens einer Sprecherin oder einem Sprecher vertreten werden.

Der Fachschaftsrat wählt eine Sprecherin oder einen Sprecher für Verwaltung, eine Sprecherin oder einen Sprecher für
Finanzen und eine Sprecherin oder einen Sprecher für Öffentlichkeitsarbeit.

Für das Amt als Sprecherin oder Sprecher können sich nur Studierende aus dem Kreis der gewählten Mitglieder des
Fachschaftsrates aufstellen lassen.

\Clause{title={Wahlen}}

'S Das Amt einer Sprecherin oder eines Sprechers wird mit absoluter Mehrheit der stimmberechtigten Mitglieder gewählt'.
Die Ämter werden einzeln und getrennt voneinander gewählt.

Sollte im einem Wahlgang mit mehreren Kandidatinnen oder Kandidaten keine absolute Mehrheit der stimmberechtigten
Mitglieder für eine Kandidatin oder einen Kandidaten zu Stande kommen, ist ein weiterer Wahlgang für dieses Amt
durchzuführen, wobei die Kandidatin oder der Kandidat mit den wenigsten Stimmen nicht mehr zur Wahl steht.

'S Das Amt als Sprecherin oder Sprecher wird für eine Wahlperiode gewählt'. Eine Wiederwahl ist möglich'. Die
Wahlperiode endet außerdem durch:
\begin{enumerate}[\qquad a)]
	\item Rücktritt;
	\item Austritt aus der Studierendenschaft;
	\item Wechsel der Fachschaft;
	\item Exmatrikulation;
	\item Konstruktiven Misstrauensantrag.
\end{enumerate}

Ist ein Amt wegen vorzeitiger Beendigung der Amtszeit neu zu besetzen, wird dieses Amt für die laufende Wahlperiode neu 
gewählt.

\Clause{title={Aufgaben und Rechte des Vorstands}}

Der Vorstand vertritt die Fachschaft gegenüber:
\begin{enumerate}[\qquad a)]
	\item staatlichen und gesellschaftlichen Institutionen;
	\item der Otto-von-Guericke-Universität Magdeburg;
	\item den Organen der Universitätsverwaltung;
	\item sowie der allgemeinen Öffentlichkeit.
\end{enumerate}

'S Der Vorstand kann im Zeitraum zwischen zwei Sitzungen im Rahmen seines operativen Geschäfts über ein Gesamtbudget
i.~H.~v. 500,00~€ während eines Haushaltsjahres verfügen'. Hierfür müssen mindestens 2 Sprecherinnen und Sprecher ihre
Zustimmung schriftlich oder in elektronischer Form geben'. Dies betrifft insbesondere die Aufrechterhaltung des
Bürobetriebs und die Vor- \& Nachbereitung der Sitzungen'. Der Verfügungsrahmen ist auch dann nicht zu überschreiten,
wenn mehrere Ausgaben in einem direkten sachlichen Zusammenhang stehen'. Der Vorstand hat dem Fachschaftsrat auf der
nächsten Sitzung Bericht zu erstatten'. Die Ausgaben werden im Protokoll festgehalten.

Der Vorstand hat eine zusätzliche Sitzung zum frühest zulässigen Termin einzuberufen (es gilt \ref{einberufung}), wenn
dies von mindestens 4 Mitglieder des Fachschaftsrates oder wenigstens einer Sprecherin oder einem Sprecher schriftlich
oder in elektronischer Form verlangt wird.

Der Vorstand hat die Möglichkeit schriftlich oder in elektronischer Form einen Umlaufbeschluss des Fachschaftsrates
einzuholen.

Die Sprecherin oder der Sprecher für Verwaltung hat insbesondere die folgenden Aufgaben:
\begin{enumerate}[\qquad a)]
	\item Vor- \& Nachbereitung der Sitzungen;
	\item Gewährleistung eines reibungslosen Bürobetriebes.
\end{enumerate}

Die Sprecherin oder der Sprecher für Finanzen hat insbesondere die folgenden Aufgaben:
\begin{enumerate}[\qquad a)]
	\item Haushalts- \& Wirtschaftsführung im Rahmen der gesetzlichen Bestimmungen;
	\item Aufstellung des Haushaltsplanes und etwaiger Nachtragshaushaltspläne.
\end{enumerate}

Die Sprecherin oder der Sprecher für Öffentlichkeitsarbeit hat insbesondere die folgenden Aufgaben:
\begin{enumerate}[\qquad a)]
	\item Formulierung von Stellungnahmen und ähnlichen öffentlichkeitswirksamen Texten;
	\item Steuerung der öffentlichen Wahrnehmung insbesondere durch Verwendung medialer Plattformen.
\end{enumerate}

\Clause{title={Ehrenmitglieder}}

'S Als Ehrenmitglieder werden passive Mitglieder bezeichnet, welche mindestens eine Wahlperiode aktives Mitglied des
Fachschaftrates waren'. Ehrenmitglieder übernehmen keinen aktiven Dienst im Fachschaftsrat und sind nicht
stimmberechtigt.

'S Zum Ehrenmitglied können aktive Mitglieder mit \nicefrac{2}{3}-Mehrheit der stimmberechtigten Mitglieder ernannt
werden'. Ehrenmitglieder werden für unbestimmte Zeit berufen'. Die Mitgliedschaft endet durch Austritt, Ausschluss oder
Tod des Mitglieds'. Der Austritt muss in Textform an den Vorstand mitgeteilt werden'. Der Ausschluss erfolgt mit
einfacher Mehrheit der stimmberechtigten Mitglieder.

'S Ehrenmitglieder dürfen in die internen Kommunikation des Fachschaftsrates eingebunden werden und Zugriff auf die
internen Dienste und Anwendungen erhalten'. Das Ehrenmitglied hat dazu einen Antrag an eine Sprecherin oder einen
Sprecher zu richten, welcher mit einfacher Mehrheit durch den gesamten Vorstand beschlossen wird'. Der Antrag ist für
jede Wahlperiode erneut zu stellen.

\end{contract}

\section{Sitzungen}

\begin{contract}

\Clause{title={Einberufung}} \label{einberufung}

'S Die Sitzungen des Fachschaftsrates finden i.~d.~R. alle 2 Wochen statt'. In der vorlesungsfreien Zeit kann von 
dieser Regelung abgesehen werden.

Die Ladungsfrist beträgt 3 Tage.

Die Einberufung erfolgt durch den Vorstand, i.~d.~R. durch die Sprecherin oder den Sprecher für Verwaltung.

'S Die Einladung zur Sitzung hat mindestens zu enthalten:
\begin{enumerate}[\qquad a)]
	\item Datum und Uhrzeit der Sitzung;
	\item Ort der Sitzung;
	\item Vorschlag der Tagesordnung.
	\setcounter{enumitem}{\value{enumi}}
\end{enumerate}
'S Ferner sollten auch enthalten sein:
\begin{enumerate}[\qquad a)]
	\setcounter{enumi}{\value{enumitem}}
	\item Anträge, die nicht persönlicher Natur sind;
	\item Berichte.
\end{enumerate}

Die Einladung ist an alle aktiven Mitglieder des Fachschaftsrates sowie an Antragstellerinnen und Antragsteller als auch
bekannte Gäste in Textform per E-Mail zu verschicken.

\Clause{title={Öffentlichkeit}}

Die Sitzungen des Fachschaftsrates sind i.~d.~R. öffentlich.

'S Der Fachschaftsrat kann mit \nicefrac{2}{3}-Mehrheit der stimmberechtigten Mitglieder für einzelne
Tagesordnungspunkte den Ausschluss der Öffentlichkeit beschließen'. Antragstellerinnen und Antragsteller können während
der Behandlung ihres Antrages um den Ausschluss der Öffentlichkeit bitten, welches mit absoluter Mehrheit der
stimmberechtigten Mitglieder beschlossen wird.

Anträge persönlicher Natur werden stets unter Ausschluss der Öffentlichkeit behandelt.

Der Fachschaftsrat kann mit absoluter Mehrheit der stimmberechtigten Mitglieder zusätzliche beteiligte oder beratende
Personen zu Teilen der Sitzung hinzuziehen, welche unter Ausschluss der Öffentlichkeit stattfinden.

Alle anwesenden Personen, die während des Ausschlusses der Öffentlichkeit an der Sitzung teilgenommen haben, sind 
zur Verschwiegenheit verpflichtet.\footnote{Die Verletzung der Schweigepflicht ist strafbar.}

\Clause{title={Protokoll}}

Über die Sitzung wird ein Protokoll geführt.

Inhalt des Protokolls sind:
\begin{enumerate}[\qquad a)]
	\item Zeitpunkt und Ort;
	\item Redeleitung, Protokollantin bzw. Protokollant sowie anwesende Mitglieder und Gäste;
	\item Abstimmungen, Beschlüsse, Wahlen und die jeweiligen Ergebnisse;
	\item Berichte von Sprecherinnen und Sprechern wie auch bei Bedarf von weiteren Mitgliedern;
	\item sowie durch die Geschäftsordnung, Finanzordnung oder Satzung vorgeschriebenen Anzeigen.
\end{enumerate}

Das nachbereitete Protokoll ist zeitnah über die internen Strukturen des Fachschaftsrates den aktiven Mitgliedern
zugänglich zu machen, spätestens jedoch bis zum Zeitpunkt der Einladung zur nachfolgenden Sitzung.

'S Das Protokoll gilt als beschlossen wenn auf jener Sitzung, zu welcher das nachbereitete Protokoll den aktiven
Mitgliedern zugänglich gemacht wurde, kein Widerspruch durch ein aktives Mitglied erhoben wird'. Im Falle eines
Widerspruches sind die Änderungsvorschläge des widersprechenden Mitgliedes als Änderungsanträge zu behandeln und
anschließend das Protokoll mit den angenommenen Änderungen durch den Fachschaftsrat zu beschließen.

'S Öffentliche Teile eines beschlossenen Protokolls sind der Studierendenschaft zugänglich zu machen'. Wurde ein
Protokoll durch den Fachschaftsrat abgelehnt ist über das weitere Verfahren durch den Fachschaftsrat zu beraten'. Die
Beschlüsse, auch aus Teilen der Sitzung, welche unter Ausschluss der Öffentlichkeit stattgefunden haben, sind in jedem
Fall (anonymisiert) zu veröffentlichen.

\Clause{title={Beschlussfassung}}

Der Fachschaftsrat ist beschlussfähig, wenn
\begin{enumerate}[\qquad a)]
	\item die Einberufung nach \ref{einberufung} ordnungsgemäß erfolgt ist;
	\item mehr als die Hälfte der stimmberechtigten Mitglieder anwesend ist.
\end{enumerate}

Die Beschlussfähigkeit ist zu Beginn der Sitzung im Protokoll festzuhalten.

'S Sollte der Fachschaftsrat auf zwei aufeinander folgenden Sitzungen nicht beschlussfähig sein, kann der Vorstand eine
Sitzung einberufen, in welcher der Fachschaftsrat unter Anwesenheit einer beliebigen Anzahl an stimmberechtigten
Mitglieder beschlussfähig ist'. Dies muss in der Einladung deutlich gekennzeichnet sein'. Die Einladung muss weiterhin
ordnungsgemäß erfolgen.

Der Fachschaftsrat entscheidet auf seinen Sitzungen i.~d.~R. mit einfacher Mehrheit der abgegebenen gültigen Stimmen, 
sofern durch Satzung, Finanz-, Beitrags- und Geschäftsordnung keine andere Mehrheit vorgeschrieben ist.

'S Für Umlaufbeschlüsse sind nur gewählte Mitglieder des Fachschaftsrates stimmberechtigt'. Der Umlaufbeschluss gilt als
angenommen, sobald die absolute Mehrheit der stimmberechtigten Mitglieder diesem zugestimmt hat'. Ist der
Umlaufbeschluss bis zur nächsten Sitzung noch nicht entschieden, so ist der Beschluss regulär auf der Sitzung zu
tätigen.

Die Beschlüsse des Fachschaftsrates sind bindend.

\Clause{title={Redeleitung}}

'S Die Redeleitung leitet die Sitzung'. Sie ist angehalten, ein heterogenes Meinungsbild einzuholen und eine
zielführende Diskussion zu ermöglichen'. Sie erteilt und entzieht das Wort.

\Clause{title={Anträge zur Geschäftsordnung}}

'S Anträge zur Geschäftsordnung können von jedem aktiven Mitglied gestellt werden'. Jedes aktive Mitglied besitzt bei
einer Abstimmung über einen Antrag zur Geschäftsordnung eine Stimme.

Anträge zur Geschäftsordnung sind umgehend, jedoch ohne Unterbrechung eines gegenwärtigen Wortbeitrages, zu behandeln.

'S Bei allen Anträgen zur Geschäftsordnung ist eine Gegenrede\footnote{Anstelle einer inhaltlichen Gegenrede 
ist ebenso eine formale Gegenrede für die Initiierung der Abstimmung ausreichend.} möglich'. Sollte keine Gegenrede 
erfolgen, so gilt der Antrag zur Geschäftsordnung als einstimmig angenommen.

Die Anträge zur Geschäftsordnung werden mit folgenden Mehrheiten angenommen.

\KOMAoptions{parnumber=false}

Durch Antrag eines aktiven Mitglieds:
\begin{enumerate}[\qquad a)]
	\item Namentliche Abstimmung
	\item Geheime Abstimmung
	\item Feststellung der Beschlussfähigkeit
	\item Rede zu rechtlichen Gegebenheiten
	\setcounter{enumitem}{\value{enumi}}
\end{enumerate}

Mit einfacher Mehrheit der abgegebenen Stimmen:
\begin{enumerate}[\qquad a)]
	\setcounter{enumi}{\value{enumitem}}
	\item Überweisung an einen dafür einberufenen Arbeitskreis
	\item Unterbrechung der Sitzung\footnote{Es ist die Dauer der Unterbrechung bei Antragstellung zu nennen.}
	\item Ende der Debatte und sofortige Abstimmung
	\item Schließung der Redeliste
	\item Eintritt in einen Tagesordnungspunkt
	\setcounter{enumitem}{\value{enumi}}
\end{enumerate}

Mit absoluter Mehrheit der Mitglieder:
\begin{enumerate}[\qquad a)]
	\setcounter{enumi}{\value{enumitem}}
	\item Änderung der Tagesordnung (mit Vorschlag)
	\item Behandlung unter einem späteren Tagesordnungspunkt
	\item Vertagung
	\setcounter{enumitem}{\value{enumi}}
\end{enumerate}

Mit \nicefrac{2}{3}-Mehrheit der Mitglieder:
\begin{enumerate}[\qquad a)]
	\setcounter{enumi}{\value{enumitem}}
	\item Wechsel der Redeleitung
	\item Nichtbefassung
\end{enumerate}

\KOMAoptions{parnumber=true}

'S Für den Fall, das sowohl ein Antrag auf geheime als auch namentliche Abstimmung gestellt wird, so ist zuerst über die
geheime Abstimmung abzustimmen'. Sollte die geheime Abstimmung angenommen werden, entfällt die Abstimmung über eine
namentliche Abstimmung.

Anträge zur Geschäftsordnung werden weder namentlich noch geheim abgestimmt.

\Clause{title={Anträge}}

'S Anträge sind Entwürfe zu Beschlüssen'. Diese sind vor Einberufung der Sitzung bei der Sprecherin oder dem Sprecher
für Verwaltung einzureichen und sind mit der Einladung zur verteilen.

'S Ein Antrag wird i.~d.~R. nur in Anwesenheit der Antragstellerin oder des Antragstellers behandelt'. Andernfalls wird
der Antrag auf die nächste Sitzung vertagt'. Ein Antrag wird höchstens dreimal vertagt.

'S Initiativanträge sind Anträge, die nach Ablauf der regulären Einreichungsfrist bei der Sprecherin oder dem Sprecher
für Verwaltung eingereicht wurden und nicht in regulärer Frist gestellt werden konnten oder auf einem Sachverhalt
beruhen der nach Einladung bekannt geworden ist'. Initiativanträge werden nur behandelt, wenn mindestens eine der
folgenden Bedingungen zutrifft.
\begin{enumerate}[\qquad a)]
	\item Unterzeichnung von wenigstens drei aktiven Mitgliedern
	\item Unterzeichnung von wenigstens einer Sprecherin oder einem Sprecher
	\item Befürwortung beim Beschluss der Tagesordnung
\end{enumerate}

'S Konstruktive Misstrauensanträge müssen 10 Tage vor Sitzungsbeginn schriftlich oder in elektronischer Form eingegangen
sein'. Konstruktive Misstrauensanträge gelten als bestätigt, wenn sie mit \nicefrac{2}{3}-Mehrheit der stimmberechtigten
Mitglieder beschlossen worden sind'. Misstrauensanträge sind vertraulich zu behandeln, werden geheim abgestimmt und
können nicht initiativ eingebracht werden.

'S Aktive Mitglieder des Fachschaftsrates können während einer Sitzung Änderungen an den Anträgen vorschlagen'. Ein
Änderungsantrag darf dem Zweck, Sinn sowie der Natur des ursprünglichen Antrages nicht widersprechen.

\Clause{title={Abstimmungen}}

Vor jeder Abstimmung liest die Redeleitung den Gegenstand der Abstimmung genau und neutral vor.

'S Vor der Abstimmung über einen Antrag sind alle dazu gestellten Änderungsanträge, in der Reihenfolge ihrer Tragweite,
beginnend mit dem weitest gehenden, abzustimmen'. Erst danach darf über den eigentlichen Antrag entschieden werden.

'S Anträge über die einmal abgestimmt wurde, können auf der laufenden Sitzung nicht noch einmal zur Abstimmung gestellt
werden'. Ausgenommen hiervon sind unter anderem der Haushaltsplan und andere Anträge, sofern dies in Satzung, Beitrags-,
Finanz- oder Geschäftsordnung anders geregelt ist.

\end{contract}

\section{Sonstiges}

\begin{contract}

\Clause{title={Einbeziehung von stellvertretenden Mitgliedern}}

'S Ist ein gewähltes Mitglied des Fachschaftsrates nicht in der Lage an den Sitzungen des Fachschaftsrates teilzunehmen,
so wird es durch ein stellvertretenden Mitglied für die Dauer der Sitzung vertreten'. Das stellvertretende Mitglied
ergibt sich entsprechend der Reihenfolge laut den amtlichen Wahlergebnissen'. Die Mitgliedschaft wird in diesem Falle
nicht beendet, lediglich das Stimmrecht wird für die Dauer der Sitzung abgetreten'. Die Abwesenheit ist der Sprecherin
oder dem Sprecher für Verwaltung bis zum Beginn der Sitzung mitzuteilen.

'S Ist eine Vertretung nicht rechtzeitig bestimmt, so wird zum Zeitpunkt der Feststellung einer Beschlussfähigkeit für
die gegenwärtige Sitzung ein anwesendes stellvertretendes Mitglied festgelegt, welches das Stimmrecht für die Dauer der
Sitzung wahrnimmt'. Sollte mehr als ein stellvertretendes Mitglied anwesend sein, übernimmt jenes stellvertretende
Mitglied mit den meisten Stimmen laut den amtlichen Wahlergebnissen die Vertretung.

Gibt es mehrere Listen\footnote{Als Listen werden die jeweils zur Wahl angetretenen politischen Gruppierungen eines
Gremiums bezeichnet.} im Fachschaftsrat gelten als mögliche Vertretung nur die stellvertretenden Mitglieder der
jeweiligen Liste des abwesenden Mitglieds.

Die zur aktiven Teilnahme an der Sitzung notwendigen Unterlagen werden der Vertretung zur Verfügung gestellt.

'S Sollte ein gewähltes Mitglied drei Mal in Folge unentschuldigt und ohne benannte zeitweilige Vertretung den
ordentlichen Sitzungen des Fachschaftsrates fernbleiben, verliert das Mitglied seine Mitgliedschaft'. Entsprechend wird
das Stimmrecht auf das nächste stellvertretende Mitglied der Liste entsprechend der amtlichen Wahlergebnisse
übertragen'. Darüber informiert die Sprecherin oder der Sprecher für Verwaltung den Fachschaftsrat.

\Clause{title={Änderungen der Geschäftsordnung}}

Eine Änderung der Geschäftsordnung des Fachschaftsrates wird mit der Mehrheit der abgegebenen gültigen Stimmen
beschlossen.\footnote{Die benötigte Mehrheit zur Änderung der Geschäftsordnung ergibt sich (analog für die Satzung der
Studierendenschaft) entsprechend dem Hochschulgesetz für das Land Sachsen-Anhalt. Eine Änderung jener Mehrheit ist nicht
rechtskräftig.}

Änderungen treten mit Beschluss sofort in Kraft.

\Clause{title={Schlussbestimmungen}}

Die Geschäftsordnung tritt mit Beschlussfassung in Kraft.

Die hier verwendeten Funktionsbezeichnungen gelten für alle Geschlechter.

'S Sollte eine Klausel dieser Geschäftsordnung unwirksam sein oder werden, so wird hiervon die Wirksamkeit der übrigen
Geschäftsordnung nicht berührt'. Unwirksame Klauseln sind im Wege der Auslegung zu ergänzen'. Sollte dies nicht möglich
sein, tritt an deren Stelle dispositives Gesetzesrecht.

Die Geschäftsordnung ist dem Studierendenrat anzuzeigen und von diesem zu veröffentlichen.

\end{contract}

\vspace{1cm}

Magdeburg, den 27.06.2022

\vspace{1cm}

\begin{table}[h]
	\centering
	\begin{tabular}{ccc}
		\cmidrule(lr){1-1} \cmidrule(lr){2-2} \cmidrule(lr){3-3}
		Sprecher für Verwaltung & Sprecherin für Finanzen & Spr. für Öffentlichkeitsarbeit \\
		Aiven Timptner & Paula Hünecke & Johanna Kühn
	\end{tabular}
\end{table}

\end{document}