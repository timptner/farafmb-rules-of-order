\documentclass[%
    parskip=half,
]{scrartcl}

% Deutsche Rechtschreibung
\usepackage[ngerman]{babel}

% Listen mit Buchstaben
\usepackage{enumerate}
\newcounter{enumitem}

% Kunstvolle Brüche
\usepackage{nicefrac}

\usepackage[%
    clausemark=forceboth,
    juratotoc,
    juratocnumberwidth=2.5em,
]{scrjura}

\pagestyle{myheadings}

% Auswählbare Verlinkungen
\usepackage[hidelinks]{hyperref}

\useshorthands{'}
\defineshorthand{'S}{\Sentence\ignorespaces}
\defineshorthand{'.}{. \Sentence\ignorespaces}

\begin{document}

\title{Geschäftsordnung}
\date{23.\,11.\,2020}
\author{Fachschaftsrat Maschinenbau}

\maketitle

Geschäftsordnung für den Fachschaftsrat Maschinenbau der Otto-von-Guericke-Universität Magdeburg - nachfolgend Fachschaftsrat genannt - in der Fassung vom 23.11.2020.

\tableofcontents

\clearpage

\addsec{Vorwort}

Zum besseren Verständnis der Geschäftsordnung sind hier einige Erläuterung der verwendeten Formulierungen zu finden.

\minisec{Gewählte Mitglieder}

Gewählte Mitglieder sind Mitglieder, die während der Hochschulwahlen gewählt wurden bzw. die Personen, die aufgrund der Beendigung der Mitgliedschaft eines Mitgliedes nachfolgen.
Sie sind stimmberechtigt.

\minisec{Stellvertretende Mitglieder}

Stellvertretende Mitglieder sind Mitglieder, die als Stellvertretung während der Hochschulwahlen gewählt wurden.
Sie können ein gewähltes Mitglied auf Sitzungen vertreten und erhalten hierdurch das Stimmrecht dieses Mitglieds.
Ein stellvertretendes Mitglied kann pro Sitzung nur ein gewähltes Mitglied vertreten.

Bei Beendigung der Mitgliedschaft eines gewählten Mitgliedes wird ein stellvertretendes Mitglied zum Nachfolger.
Hierbei ist die Reihenfolge entsprechend der amtlichen Wahlergebnisse zu berücksichtigten.

\minisec{Kooptierte Mitglieder}

Kooptierte Mitglieder sind Mitglieder, die durch Kooptation in den Fachschaftsrat aufgenommen wurden.
Sie sind nicht stimmberechtigt.

\minisec{Stimmberechtigte Mitglieder}

Stimmberechtigte Mitglieder sind gewählte oder stellvertretende Mitglieder, die auf einer Sitzung ein Stimmrecht besitzen.

\minisec{Textform}

Bezeichnet das Verfassen eines (physischen oder digitalen) Dokuments, worin der Verfasser erkenntlich ist.
Eine Unterschrift wird nicht benötigt.

\minisec{Elektronische Form}

Bezeichnet das Verfassen eines digitalen Dokuments mit digitaler Signatur der erstellenden Person(en).

\minisec{Schriftform / schriftlich}

Bezeichnet das Verfassen eines physischen Dokuments mit Unterschrift der erstellenden Person(en).

\clearpage

\appendix

\section{Aufbau}

\begin{contract}

\Clause{title={Anwendungsbereich}}

Die Geschäftsordnung regelt insbesondere:
\begin{enumerate}[\qquad a)]
    \item den Ablauf, die Organisation und Durchführung der Sitzungen;
    \item die Zusammensetzung;
    \item die Arbeit und interne Struktur;
\end{enumerate}
des Fachschaftsrates.

\Clause{title={Zusammensetzung}}

'S Der Fachschaftsrat besteht aus gewählten, stellvertretenden und kooptierten Mitglieder'.
Die Anzahl der gewählten Mitglieder beträgt 5'.
Jedes Mitglied des Fachschaftsrates muss auch Mitglied der Fachschaft sein'.
Die Kooptation von Mitgliedern wird mit einer \nicefrac{2}{3}-Mehrheit der stimmberechtigten Mitglieder beschlossen.

Der Fachschaftsrat teilt seine Arbeit in die folgenden Sachgebiete ein:
\begin{enumerate}[\qquad a)]
    \item Verwaltung;
    \item Finanzen;
    \item Öffentlichkeitsarbeit;
    \item Studium \& Lehre;
    \item Exkursionen;
    \item Hochschulpolitik;
    \item Mentoring;
    \item Technik;
    \item Veranstaltungen.
\end{enumerate}

\Clause{title={Sprecher:innen \& Sekretär:innen}}

'S Der Fachschaftsrat wählt aus seiner Mitte Sprecher:innen und Sekretär:innen, welche die Leitung eines Sachgebietes übernehmen'.
Jedes Sachgebiet hat nur eine:n Sprecher:in oder Sekretär:in.

'S Der Fachschaftsrat wählt eine:n Sprecher:in für Verwaltung und eine:n Sprecher:in für Finanzen'.
Die weiteren Sachgebiete können beliebig durch Sprecher:innen oder Sekretär:innen besetzt werden.

'S Als Sprecher:in können sich nur Studierende aus dem Kreis der gewählten Mitglieder des Fachschaftsrates aufstellen lassen'.
Die Sprecher:innen bilden gemeinsam den Vorstand'.
Der Fachschaftsrat kann von mindestens 2 Sprecher:innen vertreten werden.

'S Als Sprecher:in können sich nur Studierende aus dem Kreis der gewählten Mitglieder des Fachschaftsrates aufstellen lassen'.
Die Sprecher:innen bilden gemeinsam den Vorstand'.
Der Fachschaftsrat kann von mindestens 2 Sprecher:innen vertreten werden.

'S Als Sekretär:in können sich nur Studierende aus dem Kreis der stellvertretenden oder kooptierten Mitglieder des Fachschaftsrates aufstellen lassen.

\Clause{title={Wahlen}}

'S Die Leitung eines Sachgebietes wird mit absoluter Mehrheit der stimmberechtigten Mitglieder gewählt'.
Die Sachgebiete werden einzeln und getrennt voneinander gewählt.

Sollte im einem Wahlgang mit mehreren Kandidat:innen keine absolute Mehrheit der stimmberechtigten Mitglieder für eine:n Kandidat:in zu Stande kommen, ist ein weiterer Wahlgang für dieses Sachgebiet durchzuführen, wobei der:die Kandidat:in mit den wenigsten Stimmen nicht mehr zur Wahl steht.

'S Die Leitung eines Sachgebiets wird für eine Wahlperiode gewählt'. 
Eine Wiederwahl ist möglich'.
Die Wahlperiode endet außerdem durch:
\begin{enumerate}[\qquad a)]
    \item Rücktritt;
    \item Austritt aus der Studierendenschaft;
    \item Wechsel der Fachschaft;
    \item Exmatrikulation;
    \item Konstruktiven Misstrauensantrag.
\end{enumerate}

Ist ein Amt wegen vorzeitiger Beendigung der Amtszeit neu zu besetzen, wird dieses Amt für die laufende Wahlperiode neu gewählt.

\Clause{title={Aufgaben und Rechte der Sprecher:innen}}

Die Sprecher:innen vertreten die Fachschaft:
\begin{enumerate}[\qquad a)]
    \item gegenüber staatlichen und gesellschaftlichen Institutionen;
    \item der Otto-von-Guericke-Universität Magdeburg;
    \item den Organen der Universitätsverwaltung;
    \item sowie der allgemeinen Öffentlichkeit.
\end{enumerate}

'S Die Sprecher:innen können im Zeitraum zwischen zwei Sitzungen im Rahmen ihres operativen Geschäfts über ein Gesamtbudget i.~H.~v. 200,00~€ verfügen'.
Hierfür müssen mindestens 2 Sprecher:innen ihre Zustimmung schriftlich oder in elektronischer Form geben'.
Dies betrifft insbesondere die Aufrechterhaltung des Bürobetriebs und die Vor- \& Nachbereitung der Sitzungen'.
Der Verfügungsrahmen ist auch dann nicht zu überschreiten, wenn mehrere Ausgaben in einem direkten sachlichen Zusammenhang stehen'. 
Die Sprecher:innen haben dem Fachschaftsrat auf der nächsten Sitzung Bericht zu erstatten'.
Die Ausgaben werden im Protokoll festgehalten.

Die Sprecher:innen haben eine zusätzliche Sitzung zum frühest zulässigen Termin einzuberufen, wenn dies von mindestens 4 Mitglieder des Fachschaftsrates oder mindestens einem:einer Sprecher:in schriftlich oder in elektronischer Form verlangt wird.

Die Sprecher:innen haben die Möglichkeit schriftlich oder in elektronischer Form einen Umlaufbeschluss des Fachschaftsrates einzuholen.

Der:Die Sprecher:in für Verwaltung hat insbesondere die folgenden Aufgaben:
\begin{enumerate}[\qquad a)]
    \item Vor- \& Nachbereitung der Sitzungen;
    \item Gewährleistung eines reibungslosen Bürobetriebes.
\end{enumerate}

Der:Die Sprecher:in für Finanzen hat insbesondere die folgenden Aufgaben:
\begin{enumerate}[\qquad a)]
    \item Haushalts- \& Wirtschaftsführung im Rahmen der gesetzlichen Bestimmungen;
    \item Aufstellung des Haushaltsplanes und etwaiger Nachtragshaushaltspläne.
\end{enumerate}

\end{contract}

\section{Sitzungen}

\begin{contract}

\Clause{title={Einberufung}}\label{ladung}

'S Die Sitzungen des Fachschaftsrates finden i.~d.~R. alle 2 Wochen statt'.
In der vorlesungsfreien Zeit kann von dieser Regelung abgesehen werden.

Die Ladungsfrist beträgt 3 Tage.\label{ladungsfrist}

Die Einberufung erfolgt durch den Vorstand, i.~d.~R. durch den:die Sprecher:in für Verwaltung.

'S Die Einladung zur Sitzung hat mindestens:
\begin{enumerate}[\qquad a)]
    \item Datum und Zeit der Sitzung;
    \item Ort der Sitzung;
    \item Vorschlag der Tagesordnung;
    \setcounter{enumitem}{\value{enumi}}
\end{enumerate}
zu enthalten'.
Ferner sollten auch:
\begin{enumerate}[\qquad a)]
    \setcounter{enumi}{\value{enumitem}}
    \item Anträge, die nicht persönlicher Natur sind;
    \item Berichte;
\end{enumerate}
enthalten sein.

Die Einladung ist an alle Mitglieder des Fachschaftsrates sowie an zum Zeitpunkt der Einladung benannte Stellvertreter:innen, Antragsteller:innen und bekannte Gäste in Textform per E-Mail zu verschicken.

\Clause{title={Öffentlichkeit}}

Die Sitzungen des Fachschaftsrates sind i.~d.~R. öffentlich.

'S Der Fachschaftsrat kann mit \nicefrac{2}{3}-Mehrheit der stimmberechtigten Mitglieder für einzelne Tagesordnungspunkte den Ausschluss der Öffentlichkeit beschließen'.
Sollte eine antragstellende Person um Auschluss der Öffentlichkeit bitten, wird dies mit absoluter Mehrheit der anwesenden Mitglieder beschlossen.

Anträge persönlicher Natur werden stets nicht-öffentlich behandelt.

Der Fachschaftsrat kann mit absoluter Mehrheit der stimmberechtigten Mitglieder zusätzliche beteiligte oder beratende Personen zu nicht-öffentlichen Teilen der Sitzung hinzuziehen.

Über nicht-öffentliche Teile der Sitzung haben alle Beteiligten Verschwiegenheit zu bewahren.

\Clause{title={Protokoll}}

Über die Sitzung wird ein Protokoll geführt.

Inhalt des Protokolls sind:
\begin{enumerate}[\qquad a)]
    \item Ort, Sitzungsleitung, Protokollant sowie anwesende Mitglieder und Gäste;
    \item Abstimmungen, Beschlüsse und die jeweiligen Ergebnisse;
    \item Berichte von Sprecher:innen und Ratssekretär:innen;
    \item sowie durch die Geschäftsordnung, Finanzordnung oder Satzung vorgeschriebenen Anzeigen.
\end{enumerate}

Das nachbereitete Protokoll ist zeitnah über die internen Strukturen des Fachschaftsrates den Mitgliedern zugänglich zu machen, spätestens jedoch bis zum Zeitpunkt der Einladung zur nachfolgenden Sitzung.

'S Das Protokoll gilt als beschlossen wenn auf jener Sitzung, zu welcher das nachbereitete Protokoll den Mitgliedern zugänglich gemacht wurde, kein Widerspruch durch ein Mitglied erhoben wird'.
Im Falle eines Widerspruches sind die Änderungsvorschläge des widersprechenden Mitgliedes als Änderungsanträge zu behandeln und anschließend das Protokoll mit den angenommenen Änderungen durch den Fachschaftsrat zu beschließen.

'S Öffentliche Teile eines beschlossenen Protokolls sind der Studierendenschaft zugänglich zu machen'.
Wurde ein Protokoll durch den Fachschaftsrat abgelehnt ist über das weitere Verfahren durch den Fachschaftsrat zu beraten'.
Die Beschlüsse, auch aus nicht-öffentlichen Teilen der Sitzung, sind in jedem Fall (anonymisiert) zu veröffentlichen.

\Clause{title={Beschlussfassung}}

Der Fachschaftsrat ist beschlussfähig, wenn
\begin{enumerate}[\qquad a)]
    \item die Ladung ordnungsgemäß erfolgt ist; %(siehe \ref{ladung}, insb. \refParS{ladungsfrist});
    \item mehr als die Hälfte der stimmberechtigten Mitglieder anwesend ist.
\end{enumerate}

Die Beschlussfähigkeit ist zu Beginn der Sitzung festzuhalten.

'S Sollte der Fachschaftsrat auf zwei aufeinander folgenden Sitzungen nicht beschlussfähig sein, können die Sprecher:innen eine Sitzung einberufen, in welcher der Fachschaftsrat unter Anwesenheit einer beliebigen Anzahl an stimmberechtigten Mitglieder beschlussfähig ist'.
Dies muss in der Einladung deutlich gekennzeichnet sein'.
Die Einladung muss weiterhin ordnungsgemäß erfolgen.

Der Fachschaftsrat entscheidet auf seinen Sitzungen i.~d.~R. mit einfacher Mehrheit der abgegebenen gültigen Stimmen, sofern durch Satzung, Finanz-, Beitrags- und Geschäftsordnung keine andere Mehrheit vorgeschrieben ist.

'S Für Umlaufbeschlüsse sind nur gewählte Mitglieder des Fachschaftsrates stimmberechtigt'.
Der Umlaufbeschluss gilt als angenommen, sobald die absolute Mehrheit diesem zugestimmt hat'.
Ist der Umlaufbeschluss bis zur nächsten Sitzung noch nicht entschieden, so ist der Beschluss regulär auf der Sitzung zu tätigen.

Die Beschlüsse des Fachschaftsrates sind bindend.

\Clause{title={Sitzungsleitung}}

'S Die Sitzungsleitung leitet die Sitzung'.
Sie ist angehalten, ein heterogenes Meinungsbild einzuholen und eine zielführende Diskussion zu ermöglichen'.
Sie erteilt und entzieht das Wort.

\Clause{title={Anträge zur Geschäftsordnung}}

Anträge zur Geschäftsordnung sind umgehend, jedoch ohne einen Wortbeitrag zu unterbrechen, zu behandeln.

'S Bei allen Anträgen zur Geschäftsordnung ist eine Gegenrede möglich'.
Sollte keine Gegenrede erfolgen, so gilt der Antrag zur Geschäftsordnung als einstimmig angenommen.

Die Anträge zur Geschäftsordnung werden mit folgenden Mehrheiten angenommen.
Durch Antrag eines Mitglieds:
\begin{enumerate}[\qquad a)]
    \item Namentliche Abstimmung
    \item Geheime Abstimmung
    \item Feststellung der Beschlussfähigkeit
    \item Rede zu rechtlichen Gegebenheiten
    \setcounter{enumitem}{\value{enumi}}
\end{enumerate}
Mit einfacher Mehrheit der abgegebenen Stimmen:
\begin{enumerate}[\qquad a)]
    \setcounter{enumi}{\value{enumitem}}
    \item Überweisung in eine Arbeitskreis / -gruppe oder an die Leitung eines Sachbereichs
    \item Unterbrechung der Sitzung
    \item Ende der Debatte und sofortige Abstimmung
    \item Schluss mit der Redner*innen-Liste
    \item Eintritt in einen Tagesordnungspunkt
    \setcounter{enumitem}{\value{enumi}}
\end{enumerate}​
Mit absoluter Mehrheit der stimmberechtigten Mitglieder:
\begin{enumerate}[\qquad a)]
    \setcounter{enumi}{\value{enumitem}}
    \item Änderung der Tagesordnung (mit Vorschlag)
    \item Behandlung unter einem späteren Tagesordnungspunkt
    \item Vertagung
    \setcounter{enumitem}{\value{enumi}}
\end{enumerate}
Mit \nicefrac{2}{3}-Mehrheit der stimmberechtigten Mitglieder:
\begin{enumerate}[\qquad a)]
    \setcounter{enumi}{\value{enumitem}}
    \item Wechsel der Sitzungsleitung
    \item Nichtbefassung
\end{enumerate}
\ % fix formatting

'S Für den Fall, das sowohl ein Antrag auf geheime als auch namentliche Abstimmung gestellt wird, wird zuerst über die geheime Abstimmung abgestimmt'.
Sollte die geheime Abstimmung angenommen werden, entfällt eine Abstimmung über eine namentliche Abstimmung.

Anträge zur Geschäftsordnung werden nicht namentlich oder geheim abgestimmt.

\Clause{title={Anträge}}

'S Anträge sind Entwürfe zu Beschlüssen'.
Diese sind vor Einberufung der Sitzung bei dem:der Sprecher:in für Verwaltung einzureichen und sind mit der Einladung zur verteilen.

'S Ein Antrag wird i.~d.~R. nur in Anwesenheit der antragstellenden Person behandelt'.
Andernfalls wird der Antrag auf die nächste Sitzung vertagt'.
Ein Antrag wird höchstens dreimal vertagt.

'S Initiativanträge sind Anträge, die nach Ablauf der regulären Einreichungsfrist bei dem:der Sprecher:in für Verwaltung eingereicht wurden und nicht in regulärer Frist gestellt werden konnten oder auf einem Sachverhalt beruhen der nach Einladung bekannt geworden ist'.
Initiativanträge werden nur behandelt, wenn sie von mindestens drei Mitgliedern, einem:einer Sprecher:in unterschrieben oder zum Beschluss der Tagesordnung befürwortet worden sind.

'S Konstruktive Misstrauensanträge müssen 10 Tage vor Sitzungsbeginn schriftlich oder in elektronischer Form eingegangen sein'.
Konstruktive Misstrauensanträge gelten als bestätigt, wenn sie mit \nicefrac{2}{3}-Mehrheit der stimmberechtigten Mitglieder beschlossen worden sind'.
Misstrauensanträge sind vertraulich zu behandeln, werden geheim abgestimmt und können nicht initiativ eingebracht werden.

'S Mitglieder des Fachschaftsrates können während einer Sitzung Änderungen an den Anträgen vorschlagen'.
Ein Änderungsantrag darf dem Zweck, Sinn sowie der Natur des ursprünglichen Antrages nicht widersprechen.

\Clause{title={Abstimmungen}}

Vor jeder Abstimmung liest die Sitzungsleitung den Gegenstand der Abstimmung genau und neutral vor.

'S Vor der Abstimmung über einen Antrag sind alle dazu gestellten Änderungsanträge, in der Reihenfolge ihrer Tragweite, beginnend mit dem weitest gehenden, abzustimmen'.
Erst danach darf über den Hauptantrag entschieden werden.

'S Anträge über die einmal abgestimmt wurde, können auf der laufenden Sitzung nicht noch einmal zur Abstimmung gestellt werden'.
Ausgenommen hiervon sind unter anderem der Haushaltsplan und andere Anträge, sofern dies in Satzung, Beitrags-, Finanz- oder Geschäftsordnung anders geregelt ist.

\end{contract}

\section{Sonstiges}

\begin{contract}

\Clause{title={Einbeziehung von stellvertretenden Mitgliedern}}

'S Ist ein gewähltes Mitglied des Fachschaftsrates nicht in der Lage an den Sitzungen des Fachschaftsrates teilzunehmen, so wird es durch ein stellvertretenden Mitglied für die Dauer der Sitzung vertreten'.
Das stellvertretende Mitglied ergibt sich entsprechend der Reihenfolge laut den amtlichen Wahlergebnissen'.
Die Mitgliedschaft wird in diesem Falle nicht beendet, lediglich das Stimmrecht wird für die Dauer der Sitzung abgetreten'.
Die Abwesenheit ist dem:der Sprecher:in für Verwaltung bis zum Beginn der Sitzung mitzuteilen.

'S Ist eine Vertretung nicht rechtzeitig bestimmt, so wird zum Zeitpunkt der Feststellung der Beschlussfähigkeit der Sitzung ein:e anwesende:r Stellvertreter:in festgelegt, welche:r das Stimmrecht für die Dauer der Sitzung wahrnimmt'.
Sollte mehr als ein stellvertretendes Mitglied anwesend sein, übernimmt das Mitglied mit den meisten Stimmen laut den amtlichen Wahlergebnissen die Vertretung.

Gibt es mehrere Listen im Fachschaftsrat gelten als mögliche Vertretung nur die Stellvertreter:innen der jeweiligen Liste des abwesenden Mitglieds.

Die für die Sitzung notwendigen Unterlagen werden der Vertretung zur Verfügung gestellt.

'S Sollte ein gewähltes Mitglied drei Mal in Folge unentschuldigt und ohne benannte zeitweilige Vertretung den ordentlichen Sitzungen des Fachschaftsrates fernbleiben, verliert das Mitglied seine Mitgliedschaft'.
Entsprechend wird das Stimmrecht auf den:die nächste Stellvertreter:in der Liste der amtlichen Wahlergebnisse übertragen'.
Darüber informiert der:die Sprecher:in für Verwaltung den Fachschaftsrat.

\Clause{title={Änderungen der Geschäftsordnung}}

Eine Änderung der Geschäftsordnung des Fachschaftsrates wird, analog einer Satzungsänderung der Studierendenschaft, entsprechend dem Hochschulgesetz des Landes Sachsen-Anhalt, mit der Mehrheit der abgegebenen gültigen Stimmen beschlossen.

Änderungen treten mit Beschluss sofort in Kraft.

\Clause{title={Schlussbestimmungen}}

Die Geschäftsordnung tritt mit Beschlussfassung in Kraft.

Die hier verwendeten Funktionsbezeichnungen gelten für alle Geschlechter.

'S Sollte eine Klausel dieser Geschäftsordnung unwirksam sein oder werden, so wird hiervon die Wirksamkeit der übrigen Geschäftsordnung nicht berührt'.
Unwirksame Klauseln sind im Wege der Auslegung zu ergänzen'.
Sollte dies nicht möglich sein, tritt an deren Stelle dispositives Gesetzesrecht.

Die Geschäftsordnung ist dem Studierendenrat anzuzeigen und von diesem zu veröffentlichen.

\end{contract}

\vspace{1cm}

Magdeburg, den 23.11.2020

\vspace{1cm}

\rule{5cm}{.3pt} \hfill \rule{5cm}{.3pt} \\
Sprecher:in für Verwaltung \hfill Sprecher:in für Finanzen \\
Aiven Timptner \hfill Lars Pakusch \\

\end{document}